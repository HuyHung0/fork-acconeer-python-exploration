\documentclass[tikz]{standalone}

\usepackage[utf8]{inputenc}
\usepackage[T1]{fontenc}
\usepackage{cmap}
\usepackage{amsmath}
\usepackage{amssymb}
\usepackage{verbatim}
\usepackage{bm}

\renewcommand{\familydefault}{\sfdefault}
\usepackage[cm]{sfmath}

\usetikzlibrary{bending}
\usetikzlibrary{decorations.pathreplacing}
\usetikzlibrary{decorations.pathmorphing}
\usetikzlibrary{fadings}
\usetikzlibrary{shapes}
\usetikzlibrary{calc}

\definecolor{cblue}{rgb}{0.122, 0.467, 0.706}
\definecolor{corange}{rgb}{1, 0.498, 0.055}
\colorlet{lightgray}{black!20}

\begin{document}
\begin{tikzpicture}[scale = 3.0, font=\fontsize{8}{12}\selectfont]

  % Define the coordinates for the data points
  \coordinate (A) at (0,0);
  \coordinate (B) at (1,1.5);
  \coordinate (C) at (2,-0.5);


  % Draw the curve
  \draw [corange] (A) to [out=70,in=200] (B) to [out=20,in=100] (C);

  % Draw dashed lines for interpolation
  \draw [dashed] (A) -- (B);
  \draw [dashed] (B) -- (C);

  % Add label for the interpolated curve
  %\node [red, above right] at (B) {Interpolated Curve};

  % Calculate the maximum point of the interpolated curve
  \coordinate (max) at (1.23,1.54);

  % Draw a marker for the maximum point
  \fill [black] (max) circle (1pt);
  \node [black, above right] at (max) {Max};

  % Draw the data points
  \foreach \point in {A, B, C}
    \draw [fill=cblue] (\point) circle (1pt);

\end{tikzpicture}
\end{document}
